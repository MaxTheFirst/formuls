\usepackage[T2A]{fontenc}			      % кодировка
\usepackage[utf8]{inputenc}               % кодировка исходного текста
\usepackage[english, russian]{babel}   % локализация и переносы


%%% Страница 
\usepackage{extsizes} % Возможность сделать 14-й шрифт
\usepackage{geometry}  
\geometry{left=20mm,right=20mm,top=25mm,bottom=30mm} % задание полей текста


%%%  Текст
\setlength\parindent{0pt}         % Устанавливает длину красной строки 0pt
\sloppy                                        % строго соблюдать границы текста
\linespread{1.3}                           % коэффициент межстрочного интервала
\setlength{\parskip}{0.5em}                % вертик. интервал между абзацами
%\setcounter{secnumdepth}{0}                % отключение нумерации разделов
\usepackage{multicol}				          % Для текста в нескольких колонках
%\usepackage{soul}
\usepackage{soulutf8} % Модификаторы начертания


%%% Гиппер ссылки
\usepackage{hyperref}
\usepackage[usenames,dvipsnames,svgnames,table,rgb]{xcolor}
\hypersetup{				% Гиперссылки
	unicode=true,           % русские буквы в раздела PDF\\
	pdfstartview=FitH,
	pdftitle={Заголовок},   % Заголовок
	pdfauthor={Автор},      % Автор
	pdfsubject={Тема},      % Тема
	pdfcreator={Создатель}, % Создатель
	pdfproducer={Производитель}, % Производитель
	pdfkeywords={keyword1} {key2} {key3}, % Ключевые слова
	colorlinks=true,       	% false: ссылки в рамках; true: цветные ссылки
	linkcolor=blue,          % внутренние ссылки
	citecolor=green,        % на библиографию
	filecolor=magenta,      % на файлы
	urlcolor=NavyBlue,           % на URL
}


%%% Для формул
\usepackage{amsmath}          
\usepackage{amssymb}


\usepackage{amsthm}  % for theoremstyle

\theoremstyle{plain} % Это стиль по умолчанию, его можно не переопределять.
\newtheorem*{theorem}{Теорема}
\newtheorem*{prop}{Утверждение}
\newtheorem*{lemma}{Лемма}
\newtheorem*{sug}{Предположение}

\theoremstyle{definition} % "Определение"
\newtheorem*{Def}{Определение}
\newtheorem*{corollary}{Следствие}
\newtheorem{problem}{Задача}[section]

\theoremstyle{remark} % "Примечание"
\newtheorem*{nonum}{Решение}
\newtheorem*{defenition}{Def}
\newtheorem*{example}{Пример}
\newtheorem*{note}{Замечание}


%%% Работа с картинками
\usepackage{graphicx}                           % Для вставки рисунков
\graphicspath{{images/}{images2/}}        % папки с картинками
\setlength\fboxsep{3pt}                    % Отступ рамки \fbox{} от рисунка
\setlength\fboxrule{1pt}                    % Толщина линий рамки \fbox{}
\usepackage{wrapfig}                     % Обтекание рисунков текстом
\graphicspath{{images/}}                     % Путь к папке с картинками


%%% облегчение математических обозначений


\newcommand{\all}{\forall}
\newcommand{\ex}{\exists}

\newcommand{\RR}{\mathbb{R}}
\newcommand{\NN}{\mathbb{N}}
%\newcommand{\C}{\mathbb{C}}             % команда уже определена где-то)
\newcommand{\ZZ}{\mathbb{Z}}
\newcommand{\EE}{\mathbb{E}}
\newcommand{\brackets}[1]{\left({#1}\right)}      % автоматический размер скобочек
% Здесь можно добавить ваши индивидуальные сокращения  

\makeatletter
\def\@seccntformat#1{\@ifundefined{#1@cntformat}%
   {\csname the#1\endcsname\space}%    default
   {\csname #1@cntformat\endcsname}}%  enable individual control
\def\section@cntformat{\thesection.\space} % section-level
\makeatother
